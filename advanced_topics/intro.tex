% Created 2019-10-12 Sat 20:45
% Intended LaTeX compiler: pdflatex
\documentclass[11pt]{article}
\usepackage[utf8]{inputenc}
\usepackage[T1]{fontenc}
\usepackage{graphicx}
\usepackage{grffile}
\usepackage{longtable}
\usepackage{wrapfig}
\usepackage{rotating}
\usepackage[normalem]{ulem}
\usepackage{amsmath}
\usepackage{textcomp}
\usepackage{amssymb}
\usepackage{capt-of}
\usepackage{hyperref}
\author{Schwinn Saeree}
\date{\today}
\title{อัลกอริทึมคืออะไร?}
\hypersetup{
 pdfauthor={Schwinn Saeree},
 pdftitle={อัลกอริทึมคืออะไร?},
 pdfkeywords={},
 pdfsubject={},
 pdfcreator={Emacs 26.3 (Org mode 9.2.6)}, 
 pdflang={English}}
\begin{document}

\maketitle
\tableofcontents

จากนิยามสากล อัลกอริทึมคือขั้นตอนของการแก้ปัญหาโดยมีข้อมูลน้ำเข้าแล้วข้อมูลส่งออก แม้ว่าอัลกอริทึมจะดูเหมือนจะเป็นนามธรรมมาก หลายๆอย่างที่พวกเราทำในชีวิตประจำวันสามารถจัดได้ว่าเป็นอัลกอริทึมเช่นกัน ยกตัวอย่างเช่นการอาบน้ำเป็นต้น โจทย์ที่เราพยายามแก้ระหว่างอาบน้ำคือการทำให้ตนเองสะอาดโดยผ่านขั้นตอนของการฟอกสบู่แล้วก็ล้างน้ำสำหรับแต่ละส่วนของร่างกาย ขบวนการนี้มีข้อมูลนำเข้าเป็นร่างกายที่สกปรกและข้อมูลส่งออกเป็นร่างกายที่สะอาด ดังนั้นเราสามารถเรียกสิ่งที่เราทำทุกวันนี้ว่า "อัลกอริทึมอาบน้ำ" ได้แล้ว

อัลกอริทึมไม่จำเป็นต้องสื่อใช้ภาษาโค้ดภาษาใดภาษาหนึ่ง จึงสามารถให้นักเขียนโปรแกรมต่างภาษาศึกษาอัลกอริทึมเดียวกันได้อย่างง่ายดาย เมื่อผู้ใดต้องการนำอัลกอริทึมนั้นไปใช้จริง จึงจะต้องนำคำอธิบายอัลกอริทึมแปลงเข้าเป็นภาษาที่ตนเองต้องการใช้

ถ้าหากเราต้องการที่จะสื่อการทำงานของอัลกอริทึมให้คนอื่นได้รับรู้ เราก็จะต้องใช้วิธีการสื่อสารที่เหมาะสมกับสถานการณ์ที่สุดซึ่งประกอบไปด้วยวิธีการสื่ออัลกอริทึมดังต่อไปนี้

High-level description: วิธีนี้คนเรามันคุ้นเคยที่สุดเนื่องจากไม่ได้มีอะไรมากไปกว่าการบอกเล่ากันธรรมดาว่าถ้าต้องการแก้ปัญหาบางอย่างต้องทำอย่างไรบ้าง ยกตัวอย่างเช่นตำราทำกับข้าวก็ถือว่าเป็น high-level description ของอัลกอริทึมประเภทหนึ่งแล้ว วิธีนี้เหมาะสมเมื่อพูดคุยกันปากเปล่าหรือพยายามเล่าขบวนการคิดที่ใช้คร่าวๆ อย่างไรก็ตาม ถ้าหากต้องการที่จะตรวจสอบรายละเอียดเล็กๆน้อยๆที่จำเป็นต้องใช้ในการเขียนอัลกริทึมออกมาเป็นโค้ดจริงๆ จะไม่เหมาะสมมากนักเนื่องจากการสื่อสารไม่เป็นระบบระเบียบและมีโอกาสไม่ชัดเจนสูง

Flowchart: วิธีนี้ใช้กล่องหลายชนิดเพื่อบ่งบอกถึงประเภทของ operation ในแต่ละขั้นตอน และใช้ลูกศรเชื่อมระหว่างแต่ละกล่องเพื่อบ่งบอกถึงลำดับการทำงาน วิธีนี้มักจะใช้แทน higher-level description เพื่อความเข้าใจที่ง่ายยิ่งขึ้นแต่อย่างไรก็ตาม ความเข้าใจง่ายจะหายไปเมื่อเริ่มเข้าถึงรายละเอียดย่อยของอัลกอริทึม
ยกตัวอย่าง flowchart ของการใช้โคมไฟ

\url{https://upload.wikimedia.org/wikipedia/commons/thumb/9/91/LampFlowchart.svg/200px-LampFlowchart.svg.png}

Pseudocode: วิธีนี้ใช้หลักการโครงสร้างเดียวกันกับโค้ด แต่ทั้งหมดผ่านคำอธิบายแบบไม่เป็นทางการ วิธีนี้เหมาะสมเมื่อต้องการอธิบายรายละเอียดการ implement โดยที่ไม่เจาะจงภาษา ความเข้าใจง่ายจะน้อยลงกว่า flowchart กับ high-level description แต่ไม่ครุมเครื่อเท่าโค้ด ตัวอย่าง psuedocode

\url{https://res.cloudinary.com/practicaldev/image/fetch/s--\_5wT3KyX--/c\_limit\%2Cf\_auto\%2Cfl\_progressive\%2Cq\_auto\%2Cw\_880/https://cdn.hashnode.com/res/hashnode/image/upload/v1556193559488/mwN-ITK1f.png}
\end{document}
